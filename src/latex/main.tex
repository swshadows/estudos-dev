% Importando defs.tex
\documentclass[12pt, A4]{article}
\usepackage[utf8]{inputenc}
\usepackage[brazil]{babel}
\usepackage{graphicx}
\bibliographystyle{plain}

\begin{document}
% Quebra de Linha
Testando nova linha \newline
Outra linha \\
Mais uma linha\\

% Divisão
\part{Part}
\section{Sessão}
\subsection{Subsessão}
\subsubsection{Subsubsessão}
\par Novo parágrafo

%Tipo da fonte
\textrm{Texto em romano} \\
\texttt{Texto em máquina de escrever} \\
\textit{Texto em itálico} \\
\textbf{Texto em negrito} \\
\textsf{Texto sem serifa} 

% Tamanho da fonte
\tiny{Tiny} \\
\scriptsize{Script} \\
\footnotesize{Footnote} \\
\small{Small} \\
\normalsize{Normal} \\
\large{large} \\
\Large{Large} \\
\LARGE{LARGE} \\
\huge{huge} \\
\Huge{Huge} \\
\normalsize

% Alinhamento
\begin{flushright}
    Texto a direita
\end{flushright}

\begin{flushleft}
    Texto a esquerda
\end{flushleft}

\begin{center}
    Texto centralizado
\end{center}
% Listas e Tabelas
% Lista Não Ordenada
\begin{itemize}
    \item Item 1
    \item Item 2
    \item Item 3
    \item Item 4
\end{itemize}

%Lista Ordenada
\begin{enumerate}
    \item Item 1
    \item Item 2
    \item Item 3
    \item Item 4
\end{enumerate}
% Quebrando página
\newpage

% Tabela
\begin{table}[hb]
    \centering
    \begin{tabular}{|c|c|}
        \hline
        Nomes & Matriculas \\
        \hline
        José & 123456-7 \\
        \hline
    \end{tabular}
    \caption{Matriculas dos alunos}
\end{table}

%Matemática
Teorema de Pitágoras
\begin{equation}
    h^2 = c_1^2 + c_2^2
\end{equation}

$$x = \frac{-b \pm \sqrt{\Delta}}{2a}$$
com $\Delta = b^2 - 4ac$

% Operadores
$2 \cdot 2 = 4$
$2 \times 2 = 4$

$2 : 2 = 1$
$\frac{2}{2} = 1$

$2^{2y}$

$\sqrt[3]{8}=2$

% Imagens
\begin{figure}[h]
    \centering
    \includegraphics[width=0.1\textwidth]{assets/dog.png}
    \caption{Emoji de cachorro}
    \label{ref_dog}
\end{figure}

% Referências
\par Podemos ver a imagem de cachorro na Figura \ref{ref_dog}

% Bibliografia
De acordo com \cite{ibge1993} e \cite{abntex2-wiki-como-customizar}...

\bibliography{bibliografia}
\nocite{abntex2cite}

\tableofcontents
\listoffigures
\listoftables
\end{document}
